\documentclass[letterpaper,12pt]{article}
\usepackage{tabularx} % extra features for tabular environment
\usepackage{amsmath}  % improve math presentation
\usepackage{float}
\usepackage{pdfpages}


\usepackage{graphicx} % takes care of graphic including machinery
\graphicspath{ {./figures/} }
%\usepackage[margin=1in,letterpaper]{geometry} % decreases margins
%\usepackage{cite} % takes care of citations
\usepackage[final]{hyperref} % adds hyper links inside the generated pdf file
\hypersetup{
	colorlinks=true,       % false: boxed links; true: colored links
	linkcolor=blue,        % color of internal links
	citecolor=blue,        % color of links to bibliography
	filecolor=magenta,     % color of file links
	urlcolor =blue         
}
\usepackage[margin = 1in,headsep=0.5cm,headheight=2cm,letterpaper]{geometry} 

\usepackage{fancyhdr}
\pagestyle{fancy}
\lhead{Student 1 : Ahmet Akman 2442366 \\ Student 2: Yusuf Toprak Yıldıran 2444149 \\ Assistant: Onur Selim Kılıç}
\rhead{Date: \today \\ Group: Wednesday Morning - 5} 
%\cfoot{center of the footer!}
\renewcommand{\headrulewidth}{0.1pt}



\begin{document}
\thispagestyle{empty}

\title{Spring 2022 EE214 Experiment 3  \protect\\ Transformers and MATLAB Workshop }
\author{Ahmet Akman 2442366 \protect\\ Yusuf Toprak Yıldıran 2444149 \protect\\ Assistant: Onur Selim Kılıç}
\date{\today}
\maketitle
\tableofcontents
%\begin{abstract}
%abstract
%\end{abstract}
\section{Introduction}
In this experiment, transformers and MATLAB workshop, we are required to get used to plotting with MATLAB and expermented with different transformer setups. First, we are expected to observe the step-down property of the transformer. Then, the behaviour of the tranformer with a resistive load is expected to be experimented with different frequencies. The transformers characteristics are needed to be formulated with an equivalent circuit. Lastly, advantage of the impedance matching circuit setup is requested to be observed.
\section{Experimental Results and Discussion}
The results of the experiment are discussed in the following steps.
%
\subsection{Step 1}
In this step transformer circuit with no resistive load is examined. 
\begin{figure}[H]
    \centering
    \includegraphics[width = 0.75\textwidth]{1.png}
    \caption{Circuit schematic for the step 1}
\end{figure} 
    
In first step, following circuit given in Figure X is set and it is observed that step-down operation of the tansformer under no load with sinusoidal input voltage which has a peak to peak voltage of 20 V and frequency of
50 Hz. Then, \(V_in(t) \) and \(V_out(t)\) are plotted on the graph in Figure X.
\begin{figure}[H]
    \centering
    \includegraphics[width = 0.75\textwidth]{1_1.png}
    \caption{\(V_in(t) \) and \(V_out(t)\) vs Time}
\end{figure} 
Afterwards, \(N_1:N_2\) ratio is measured as \(19.5/2.5 \approx 7.8 \).

%
\subsection{Step 2}
In this step transformer circuit with a resistive load is examined.
\begin{figure}[H]
    \centering
    \includegraphics[width = 0.75\textwidth]{2.png}
    \caption{Circuit schematic for the step 2}
\end{figure} 

For this step, signal generator output is adjusted as \(V_s = 10sin(2\pi50t)\) with 20V peak to peak voltage and 50Hz frequency ;then, Transformer circuit with resistive load is constructed as given in the Figure X where R = \(56\Omega \).  
\subsubsection{i}

In this step, to obtain current \(I_in\), \(1K\Omega \) resistor is connected between - terminal of primary side transformer and - terminal of signal generator. Then, CH1 is connected to + terminal of the signal generator and CH2 probe is to the - terminal of primary side transformer. By subtracting CH2 from CH1 \(V_in\) is obtained, and CH2 probe of DSO gives \(I_in\) in mA. Afterwards, \(V_in\) and \(I_in\) are plotted in Figure X. Then, \(V_rms\) and \(I_rms\) are obtained using DSO measurement tool as 3.5V and 3.7mA respectively given in figure X. 
\begin{figure}[H]
    \centering
    \includegraphics[width = 0.75\textwidth]{2_1.png}
    \caption{\(V_in(t) \) and \(I_in(t)\) vs Time}
\end{figure} 

\begin{table}[H]
    \begin{center}
        \caption{RMS Values of input \(V_rms\) and input \(I_rms\)}
        \vspace{2mm}
        \begin{tabular}{||c | c ||} 
            \hline
            \(V_rms\) & \(I_rms\) \\ [0.5ex] 
            \hline\hline
            3.5 V & 3.7 mA    \\
            \hline
        \end{tabular}
    \end{center}
\end{table}



\subsubsection{ii}
For this step CH1 is connected to + terminal of secondary side of the transformer , \(V_out(t)\) is plotted 
\begin{figure}[H]
    \centering
    \includegraphics[width = 0.75\textwidth]{2_2.png}
    \caption{\(V_o(t) \) vs Time}
\end{figure} 

\subsubsection{iii}
\begin{figure}[H]
    \centering
    \includegraphics[width = 0.75\textwidth]{2_3_1.png}
    \caption{\(V_in(t) \) and \(I_in(t)\) vs Time (500Hz)}
\end{figure} 

\begin{figure}[H]
    \centering
    \includegraphics[width = 0.75\textwidth]{2_3_2.png}
    \caption{\(V_o(t) \) vs Time (500Hz)}
\end{figure} 
%
\subsection{Step 3}
In this step transformer equivalent circuit is formulated by making the secondary terminal of the transformer short circuited.
\subsubsection{i}
By connecting a 1\(\Omega\) resistor series to the primary terminal we have obtained the plot given in Figure X which shows \(V_in(t) \) and \(I_in(t)\) on the same graph.  
\begin{figure}[H]
    \centering
    \includegraphics[width = 0.75\textwidth]{3_1.png}
    \caption{\(V_in(t) \) and \(I_in(t)\) vs Time }
\end{figure} 

\subsubsection{ii}
To obtain the equivalent circuit parameters given in Figure X. Bunch of assumptions ,and calculations are made.
\begin{figure}[H]
    \centering
    \includegraphics[width = 0.75\textwidth]{2.png}
    \caption{Transformer equivalent circuit}
\end{figure} 
From the measurement feature of the oscilloscope, \(V_{sc-RMS}\) ,and \(I_{sc-RMS}\) are found 0.3345 Volts and 0.0230 Amps respectively. So the impedance becomes;
\[
    Z = \frac{V_{sc-RMS}}{I_{sc-RMS}} = 15 \Omega
    \]
The primary side of the resistance and inductance are taken equal to the reflected equivalent secondary side resistance and inductance. The \(\theta\) , phase difference is found as 23.9326 degrees. The resistance becomes;
\[
    R = Z cos (\theta) = 13.7103
    \]
    The reactance becomes;
    \[
    X = Z sin (\theta) = 6.0849 
    \]
So as a result the following parameters are obtained by equating the overall resistance to R and overall reactance to X. From Step 1 n is taken as 8.
\begin{table}[H]
    \begin{center}
        \caption{Transformer equivalent circuit parameters}
        \vspace{2mm}
        \begin{tabular}{||c | c | c | c ||} 
            \hline
            \(r_1 (\Omega) \) & \(r_2 (\Omega)\)  & \(L_1\) (H) & \(L_2\) (H) \\ [0.5ex] 
            \hline\hline
            6.8553 & 0.1071 & 3.0425 & 0.0475 \\ 
            \hline
        \end{tabular}
    \end{center}
    \end{table}
    

    %
\subsection{Step 4}
\begin{figure}[H]
    \centering
    \includegraphics[width = 0.75\textwidth]{3.png}
    \caption{Impedance matching}
\end{figure} 

\subsubsection{i}
\subsubsection{ii}


\section{Conclusion}
In this experiment, transformers and MATLAB workshop, we got used to plotting with MATLAB and expermented with different transformer setups. First, we have observed the step-down property of the transformer. Then, the behaviour of the tranformer with a resistive load is experimented with different frequencies. The transformers characteristics are formulated with an equivalent circuit. Lastly, advantage of the impedance matching circuit setup is observed.
\section*{Appendix A}
\begin{itemize}
    \item PreLab Preparation 1 hours
    \item Experimental Work 2  hours
    \item Report Writing 5 hours
\end{itemize}

\end{document}

%%%%%%%%%%%%%%%%%%%%%%   EXAMPLE TABLE   %%%%%%%%%%%%%%%%%%%%%%%%%%%%%%%%
\begin{table}[H]
\begin{center}
    \caption{Resistance reading by color code convention.}
    \vspace{2mm}
    \begin{tabular}{||c | c | c||} 
        \hline
        Color Order & Value & Tolerance \\ [0.5ex] 
        \hline\hline
        Brown / Black / Red / Gold & 1k\( \Omega \) & \( \% \) 5  \\ 
        \hline
        Yellow / Violet / Red / Gold & 4.7k\( \Omega \) & \( \% \) 5   \\
        \hline
        Brown / Grey / Orange / Gold & 18k\( \Omega \) & \( \% \) 5  \\ [1ex] 
        \hline
    \end{tabular}
\end{center}
\end{table}


%%%%%%%%%%%%%%%%%%%%%%   EXAMPLE IMAGE   %%%%%%%%%%%%%%%%%%%%%%%%%%%%%%%%
\begin{figure}[H]
\centering
\includegraphics[width = 0.75\textwidth]{5.png}
\caption{Circuit schematic for the step 5}
\end{figure} 

%%%%%%%%%%%%%%%%%%%%%%   EXAMPLE IMAGE FROM PDF   %%%%%%%%%%%%%%%%%%%%%%%%%%%%%%%%
\begin{figure}[H] \centering{
	\includegraphics[scale=0.25]{2a_plot.pdf}}
	\caption{Experiment 2}
\end{figure}
	