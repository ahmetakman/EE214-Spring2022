\documentclass[letterpaper,12pt]{article}
\usepackage{tabularx} % extra features for tabular environment
\usepackage{amsmath}  % improve math presentation
\usepackage{float}
\usepackage{pdfpages}


\usepackage{graphicx} % takes care of graphic including machinery
\graphicspath{ {./figures/} }
%\usepackage[margin=1in,letterpaper]{geometry} % decreases margins
%\usepackage{cite} % takes care of citations
\usepackage[final]{hyperref} % adds hyper links inside the generated pdf file
\hypersetup{
	colorlinks=true,       % false: boxed links; true: colored links
	linkcolor=blue,        % color of internal links
	citecolor=blue,        % color of links to bibliography
	filecolor=magenta,     % color of file links
	urlcolor =blue         
}
\usepackage[margin = 1in,headsep=0.5cm,headheight=2cm,letterpaper]{geometry} 

\usepackage{fancyhdr}
\pagestyle{fancy}
\lhead{Student 1 : Ahmet Akman 2442366 \\ Student 2: Yusuf Toprak Yıldıran 2444149 \\ Assistant: Onur Selim Kılıç}
\rhead{Date: \today \\ Group: Wednesday Morning - 5} 
%\cfoot{center of the footer!}
\renewcommand{\headrulewidth}{0.1pt}
%

%\renewcommand{\footrulewidth}{0.4pt}



\begin{document}
\thispagestyle{empty}

\title{Spring 2022 EE214 Experiment 2  \protect\\ Miscellaneous Op-Amp Circuits}
\author{Ahmet Akman 2442366 \protect\\ Yusuf Toprak Yıldıran 2444149 \protect\\ Assistant: Onur Selim Kılıç}
\date{\today}
\maketitle
\tableofcontents
%\begin{abstract}
%abstract
%\end{abstract}
\section{Introduction}
In this experiment, miscellaneous op-amp circuits, three different setups of op-amp circuity are investigated. First, an independent current source circuit is set and its behavior is required to be characterized. Then the clipper circuit is constructed, and the output is needed to be observed. Lastly, a negative resistance converter with two zener is built with two different setups. First, its i-v characteristics are expected to be observed, then a square wave generator is expected to be set.
\section{Experimental Results and Discussion}
The results of the experiment are discussed in the following steps.
\subsection{Step 1}
In this step independent current source circuit given in Figure 1 is constructed. A potentiometer with 10K\(\Omega\) is connected to the one port as \(R_L\).
\begin{figure}[H]
    \centering
    \includegraphics[width = 0.75\textwidth]{1SCH.png}
    \caption{Circuit schematic for the step 1}
\end{figure} 
To be able to obtain the maximum value of the resistance in which the one port still functions as a independent current source, the potentiometer is meticulously adjusted. So , the parameters given in Figure 1 is obtained.
\begin{table}[H]
    \begin{center}
        \caption{Resistance reading by color code convention.}
        \vspace{2mm}
        \begin{tabular}{||c | c ||} 
            \hline
            The Current Value & Corresponding Resistance \\ [0.5ex] 
            \hline\hline
            1.24 mA & 8k\(\Omega\)    \\
            \hline
        \end{tabular}
    \end{center}
\end{table}

\subsection{Step 2}
In this step the 

\begin{figure}[H]
    \centering
    \includegraphics[width = 0.75\textwidth]{2SCH.png}
    \caption{Circuit schematic for the step 2}
\end{figure} 

\subsection{Step 3}

In this part we set the negative resistance converter circuit given in the figure X below.
\begin{figure}[H]
    \centering
    \includegraphics[width = 0.75\textwidth]{3SCH.png}
    \caption{Circuit schematic for the step 3}
\end{figure} 

  
\subsubsection{a)}
For this part a, we used 1.2k\(\Omega\) instead of the \(R_2\) 2.7k\(\Omega\) pot and adjusted the V as \(V(t) = 10sin(\pi t)V\) and obtained the i vs v characteristics by using DSO in X-Y mode. To obtain current i, we connceted 1k\(\Omega\) resistor between common ground and non-inverting input port of the op-amp and measured the voltage accross it, by doing this, we get the current in mA. Also voltage v is obtained by measuring the input signal. From oscilloscope it can be seen that op-amp goes into + saturation or - saturation without being into linear region and circuit is unstable.

\subsubsection{b)}
In this subsection b. a 1\(\mu F\) capacitor is connected accross the terminals between this one port circuit and \(R_2 \) and \(R_4 \) are adjusted untill \(V_0(t)\) become a square wave of 2 volts peak-to-peak with frequency of 500Hz. Then, \(R_2 \) and \(R_4 \) are recorded as table which is given in the figure X. It can be observed that experimental values are consistent with the preliminary work results given in figure X.


\begin{table}[H]
    \begin{center}
        \caption{Resistance reading by color code convention.}
        \vspace{2mm}
        \begin{tabular}{||c | c | c ||} 
            \hline
            \(R_2\) & \(R_4\) &  \(\tau \) \\ [0.5ex] 
            \hline\hline
            0.46k\(\Omega\) & 2.4k\(\Omega\) &  1.8x\(10^-6\)s \\
            \hline
        \end{tabular}
    \end{center}
\end{table}

\section{Conclusion}

helll
rfgtgtgtgt
\section*{Appendix A}
\begin{itemize}
    \item PreLab Preparation 4 hours
    \item Experimental Work 2  hours
    \item Report Writing 4 hours
\end{itemize}

\end{document}

%%%%%%%%%%%%%%%%%%%%%%   EXAMPLE TABLE   %%%%%%%%%%%%%%%%%%%%%%%%%%%%%%%%
\begin{table}[H]
\begin{center}
    \caption{Resistance reading by color code convention.}
    \vspace{2mm}
    \begin{tabular}{||c | c | c||} 
        \hline
        Color Order & Value & Tolerance \\ [0.5ex] 
        \hline\hline
        Brown / Black / Red / Gold & 1k\( \Omega \) & \( \% \) 5  \\ 
        \hline
        Yellow / Violet / Red / Gold & 4.7k\( \Omega \) & \( \% \) 5   \\
        \hline
        Brown / Grey / Orange / Gold & 18k\( \Omega \) & \( \% \) 5  \\ [1ex] 
        \hline
    \end{tabular}
\end{center}
\end{table}


%%%%%%%%%%%%%%%%%%%%%%   EXAMPLE IMAGE   %%%%%%%%%%%%%%%%%%%%%%%%%%%%%%%%
\begin{figure}[H]
\centering
\includegraphics[width = 0.75\textwidth]{5.png}
\caption{Circuit schematic for the step 5}
\end{figure} 

%%%%%%%%%%%%%%%%%%%%%%   EXAMPLE IMAGE FROM PDF   %%%%%%%%%%%%%%%%%%%%%%%%%%%%%%%%
\begin{figure}[H] \centering{
	\includegraphics[scale=0.25]{2a_plot.pdf}}
	\caption{Experiment 2}
\end{figure}
	