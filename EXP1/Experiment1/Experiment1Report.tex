\documentclass[letterpaper,12pt]{article}
\usepackage{tabularx} % extra features for tabular environment
\usepackage{amsmath}  % improve math presentation
\usepackage{float}
\usepackage{pdfpages}


\usepackage{graphicx} % takes care of graphic including machinery
\graphicspath{ {./figures/} }
%\usepackage[margin=1in,letterpaper]{geometry} % decreases margins
%\usepackage{cite} % takes care of citations
\usepackage[final]{hyperref} % adds hyper links inside the generated pdf file
\hypersetup{
	colorlinks=true,       % false: boxed links; true: colored links
	linkcolor=blue,        % color of internal links
	citecolor=blue,        % color of links to bibliography
	filecolor=magenta,     % color of file links
	urlcolor =blue         
}
\usepackage[margin = 1in,headsep=0.5cm,headheight=2cm,letterpaper]{geometry} 

\usepackage{fancyhdr}
\pagestyle{fancy}
\lhead{Student 1 : Ahmet Akman 2442366 \\ Student 2: Yusuf Toprak Yıldıran 2444149 \\ Assistant: Onur Selim Kılıç}
\rhead{Data: \today \\ Group: Wednesday Morning - 5}
%\cfoot{center of the footer!}
\renewcommand{\headrulewidth}{0.4pt}
%

\renewcommand{\footrulewidth}{0.4pt}


\begin{document}
\thispagestyle{empty}

\title{Spring 2022 EE214 Experiment 1  \protect\\ Diodes and Rectifiers}
\author{Ahmet Akman 2442366 \protect\\ Yusuf Toprak Yıldıran 2444149}
\date{\today}
\maketitle
\tableofcontents
%\begin{abstract}
%abstract
%\end{abstract}
\section{Introduction}
In this experiment ,  characteristics of different  diodes ,and rectifiers are investigated. First the i-v characteristics of 3 different diodes are expected to be oobserved. Then, the behavior of the half wave is expected to be experimented. Lastly, observations are made on clamper and  zener regulator circuits. The results of the experimentation is presented in this document.
\section{Experimental Results and Discussion}
The results of the experiment are discussed in following steps.
\subsection{Step 1}
In this step, the circuit schematic given in Figure TODO is constructed on breadboard. As the signal supply, analog signal generator is used for floating output.
\begin{figure}[H]
\centering
\includegraphics[width=1\textwidth]{1_1.png}
\caption{Circuit schematic for the step 1}
\end{figure} 

\subsubsection{a)}
%\href{https://www.vishay.com/docs/88503/1n4001.pdf}{1N40007} 
The diode models, \href{https://logosfoundation.org/elektron/mixers/AA119.pdf}{AA119},href{https://www.vishay.com/docs/88536/ba157.pdf}{BA159}  ,and \href{https://www.vishay.com/docs/85604/bzx55.pdf}{BZX55C-6V2} are used. The probes of the oscilloscope is connected to the nodes indicated in Figure TODO. The resulting graph is plotted as given in Figure TODO2 TODO3 ,and TODO 4 for 
\href{https://logosfoundation.org/elektron/mixers/AA119.pdf}{AA119},href{https://www.vishay.com/docs/88536/ba157.pdf}{BA159}  ,and \href{https://www.vishay.com/docs/85604/bzx55.pdf}{BZX55C-6V2} respectively.

\begin{figure}[H]
    \centering
    \includegraphics[width=1\textwidth]{1_aa119.png}
    \caption{i-v characteristics of AA119}
\end{figure} 


\begin{figure}[H]
    \centering
    \includegraphics[width=1\textwidth]{1_ba159.png}
    \caption{i-v characteristics of BA159}
\end{figure} 


\begin{figure}[H]
    \centering
    \includegraphics[width=1\textwidth]{1_zener.png}
    \caption{i-v characteristics of BZX55C-6V2}
\end{figure} 


Using those plots the piecewise parameters of the diodes are obtained by the virtue of the cursors of the oscilloscope. The parameters of diode AA119 is given in Table 1.

\begin{table}[H]
    \begin{center}
    \caption{Piecewise parameters of diode AA119}
    \vspace{2mm}
    \begin{tabular}{|| c | c | c ||}
    \hline
    \(V_{on}\) & 350 mV \\
    \hline 
    \(r_f\) & 0.17 \(\Omega\) \\
    \hline
    \(r_r\) & 86 \(\Omega\)\\
    \hline
    \end{tabular}
\end{center}
\end{table}
The obtained parameters of diode AA119 is given in Table 2.
\begin{table}[H]
    \centering
    \caption{Piecewise parameters of diode BA159}
    \vspace{2mm}
    \begin{tabular}{||c | c | c||}
    \hline
    \(V_{on}\) & 973.5 mV \\
    \hline
    \(r_f\) & 0.625 \(\Omega\) \\
    \hline
    \(r_r\) & 86.2 \(\Omega\)\\
    \hline
    \end{tabular}
\end{table}
The obtained parameters of diode BZX55C-6V2 is given in Table 3.
\begin{table}[H]
    \centering
    \caption{Piecewise parameters of diode BZX55C-6V2}
    \begin{tabular}{||c | c | c||}
        \hline
    \(V_{on}\) & 752 mV \\
    \hline
    \(V_{z}\) & 5.92V \\
    \hline
    \(r_f\) & 0.05 \(\Omega\) \\
    \hline
    \(r_r\) & 0.156 \(\Omega\) \\
    \hline
    \end{tabular}
\end{table}
So, the pimple i-v characteristics of 3 different diodes are obtained ,and analyzed using the plot.

\subsubsection{b)}


\begin{figure}[H]
    \centering
    \includegraphics[width=1\textwidth]{1_ba159_10khz.png}
    \caption{i-v characteristics of BA159 at 10khz}
\end{figure} 

We can call separation of lines in i-v graph as hysteresis effect. When we increase frequency to 10kHz, we observed hysteresis effect on the i-v characteristics of diode on the DSO screen. Since our diode can not change its state as fast as our high frequency voltage source supply, we observe this effect. To solve this problem and to be able to use diodes at high frequencies, there are "high speed" or "switching" diodes. These diodes can be used in high frequencies without observing hysteresis effect.

\subsection{Step 2}

\begin{figure}[H]
    \centering
    \includegraphics[width=1\textwidth]{2_1.png}
    \caption{Circuit schematic for the step 2}
\end{figure} 
For this part, we set the half-wave rectifier circuit given in Figure X and adjusted 
\(
v_i(t)  = 2sin(2000 \pi t) V 
\)


\subsubsection{a)}

\begin{figure}[H]
    \centering
    \includegraphics[width=1\textwidth]{2_a.png}
    \caption{Half-wave rectifier with BA159}
\end{figure} 






After setting POT to \(1k\Omega\) , we obtain output and input voltage waveforms on the graph indicaded in Figure X.


If we look at the graph carefully, we can realized that output voltage is slightly lower than input voltage since diode consumes energy. Then we measured the \(DC_{rms} \) of output value as 730mV using cursors of DSO. Output voltage waveform on the DSO screen is like same with the input waveform in positive cycles but zero when input is in the negative cycle.  


\subsubsection{b)}

\begin{figure}[H]
    \centering
    \includegraphics[width=1\textwidth]{2_b_POT_1_2K.png}
    \caption{Half-wave rectifier with BA159 with pot set to 1.2K \(\Omega\)}
\end{figure} 


\begin{figure}[H]
    \centering
    \includegraphics[width=1\textwidth]{2_b_POT_10K.png}
    \caption{Half-wave rectifier with BA159 with pot set to 10K \(\Omega\)}
\end{figure} 

\begin{figure}[H]
    \centering
    \includegraphics[width=1\textwidth]{2_b_POT_18K.png}
    \caption{Half-wave rectifier with BA159 with pot set to 18K \(\Omega\)}
\end{figure} 


\subsection{Step 3}

In this third step, we set up the diode clamper circuit given in Figure X using the diode 1N4001 and set the voltage \(v_i(t) = 10sin(200\pi t) V\). Then plotted the input and output voltage waveforms on the graph given in Figure X.

From above graph we can derive that output waveform is just shifted form of input waveform to below zero. Since diode prevent capacitor from discharged, in order peak value of output to be clamped above zero diode should go into negative bias.

\begin{figure}[H]
    \centering
    \includegraphics[width=1\textwidth]{3_1.png}
    \caption{Circuit schematic for the step 3}
\end{figure} 
    

\begin{figure}[H]
    \centering
    \includegraphics[width=1\textwidth]{3.png}
    \caption{Clamper circuit output }
\end{figure} 
    

\subsection{Step 4}

\begin{figure}[H]
    \centering
    \includegraphics[width=1\textwidth]{4_1.png}
    \caption{Circuit schematic for the step 4}
\end{figure} 
    
    
\section{Conclusion}
asdfd
\section*{Appendix A}
\begin{itemize}
    \item PreLab Preprataion 6 hours
    \item Experimental Work 2  hours
    \item Report Writing 6 hours
\end{itemize}

\end{document}

%%%%%%%%%%%%%%%%%%%%%%   EXAMPLE TABLE   %%%%%%%%%%%%%%%%%%%%%%%%%%%%%%%%
\begin{table}[H]
\begin{center}
    \caption{Resistance reading by color code convention.}
    \vspace{2mm}
    \begin{tabular}{||c | c | c||} 
        \hline
        Color Order & Value & Tolerance \\ [0.5ex] 
        \hline\hline
        Brown / Black / Red / Gold & 1k\( \Omega \) & \( \% \) 5  \\ 
        \hline
        Yellow / Violet / Red / Gold & 4.7k\( \Omega \) & \( \% \) 5   \\
        \hline
        Brown / Grey / Orange / Gold & 18k\( \Omega \) & \( \% \) 5  \\ [1ex] 
        \hline
    \end{tabular}
\end{center}
\end{table}


%%%%%%%%%%%%%%%%%%%%%%   EXAMPLE IMAGE   %%%%%%%%%%%%%%%%%%%%%%%%%%%%%%%%
\begin{figure}[H]
\centering
\includegraphics[width=1\textwidth]{5.png}
\caption{Circuit schematic for the step 5}
\end{figure} 

%%%%%%%%%%%%%%%%%%%%%%   EXAMPLE IMAGE FROM PDF   %%%%%%%%%%%%%%%%%%%%%%%%%%%%%%%%
\begin{figure}[H] \centering{
	\includegraphics[scale=0.25]{2a_plot.pdf}}
	\caption{Experiment 2}
\end{figure}
	